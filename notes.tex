\section*{notes}

This section is for the use of the writer and should not be in any published or reviewed version of this paper. If you read this as a reviewer, please notify me of the error and ignore this section.

Found the webpage http://www.me.umn.edu/education/undergraduate/writing/How-to-write-a-Design-Report.pdf
I suggest writing the Problem definition in the introduction,
\paragraph{Introduction}
\begin{itemize}
 \item Review basic principles of pFEL operation and why (we think) it doesn't work.
 \item Technical review:
 \begin{itemize}
  \item explain a bit about the electron beam in this setup
  \item explain about available methods and why they don't work
 \end{itemize}
 \item design requirements either here or in the design chapter
\end{itemize}

\paragraph{Theory}
Is good as it is I believe

\paragraph{Design}
\begin{itemize}
 \item Design requirements if these are not in introduction
 \begin{itemize}
  \item ultra-high vacuum
  \item fits the connections (drawing)
  \item accurate up to 0.1 mm
  \item in appendix??? magnetically neutral
 \end{itemize}
 \item Overview, using line drawings (rough explanation of design)
 \item detailed description, maybe block diagram of different parts. Supported by a lot of SW drawings.
 \begin{itemize}
  \item translation system
  \item plates and holders
  \item electronics
  \item data acquisition
 \end{itemize}
 \item Short section on how the system is used. describe an operation cycle from the users point of view.
\end{itemize}

\paragraph{Calibration}
\begin{itemize}
 \item Describe prototype and setup
 \item Present results
 \item conclude that requirements are met
\end{itemize}

\paragraph{Conclusion}
\begin{itemize}
 \item Recap design requirements that can be measured
 \item Recap results from calibration
 \item conclude that requirements are met
 \item give recommendations on design (next time have this shit built-in)
 
\end{itemize}