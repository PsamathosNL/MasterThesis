\chapter{Heat simulations}
\label{sec:heatsims}
%This appendix will include the simulation Joanna did on the temperature profile of a heating point in a layer of Tungsten. The goal is to show that the Tungsten is locally heated above its melting point, ergo it cannot be used. I am not certain if I should include the results for the slanted shape where it does \textit{not} melt.
An early idea to determine the position of the electron gun was to take a conductor, divide it into segments and measure the distribution of currents coming from each segment. 
The current distribution is a direct measurement of the charge distribution in the beam, so by using this method and tactically placed segments a (destructive) beam position monitor could be built.

If an electron hits a piece of material, a number of effects take place. 
It is possible that the energy is high enough to re-emit a secondary (or multiple) electrons from the surface. This is one of the principles how avalanche photodiodes work \red{[citation?]}.
The electron enters the material and is slowed down by interactions with the bound \red{[not the free?]} electrons in the material. 
\red{elaborate}



If we ignore secondary emission \red{argumenteer} and we assum the electrons lose all their energy directly at the surface, the electron beam can be seen as a surface heat source. This heat source has a power of \red{something} and a time of \SI{10}{\micro\second}.
