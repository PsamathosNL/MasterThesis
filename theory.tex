\chapter{Electrostatic approach}
\label{sec:theory}
In this chapter, the theory of detecting an electron beam position using its (quasi-) static coulomb field on capacitive plates is presented. Firstly, the coulomb field of a charge distribution inside a coaxial metallic tube is calculated. Then, the theory of probing this field is explained followed by a quantitative description of how this can be used to detect the position of an electron beam.
%------------------------------------------------

\section{Potential field of an electron beam}
Any static charge distribution has an electric field that can be described by \cite{GriffithsEM}
\begin{equation}
\textbf{E}(\mathbf{r}) = \frac{1}{4\pi\epsilon_0}\int_\mathcal{V}\frac{\rho(\mathbf{r}^\prime)}{\rcurs^2}\hat{\brcurs}d\tau^\prime
\end{equation}
where E is the electric field strength, $\mathbf{r}$ is the position vector where the field is measured, $\epsilon_0$ is the permeability of the vacuum, $\rho(\mathbf{r^\prime})$ is the charge density at position $\mathbf{r^\prime}$, \brcurs \hspace{2pt} is the vector from $\mathbf{r^\prime}$ to $\mathbf{r}$ and $d\tau^\prime$ is an infinitesimal volume element. In principal, the integral is evaluated over all space, but since $\rho$ will be zero where no charge is present, we can confine the integral to the volume $\mathcal{V}$ where charges are present.
The corresponding potential distribution is given by
\begin{equation}
\nabla^2V(\mathbf{r})=-\frac{\rho}{\epsilon_0}
\end{equation}
%V(\mathbf{r})=\frac{-1}{4\pi\epsilon_0}\int\frac{\rho(\mathbf{r}^\prime)}{\rcurs}d\tau, incorrect
where \textit{V} is the potential. However, a reference point is required to integrate from in order to get a numerical value for \textit{V}.

As charge distribution we assume the electron beam to be an infinitely long homogeneous rod of radius $r_b$, with charge density $\rho_b$. For a constant velocity $v_b$, $\rho_b$ is given by:
\begin{equation}
\label{eq:rho1}
\rho_b = \frac{I}{\pi r_b^2v_b}
\end{equation}
where $I$ is the total current of the electron beam.
If the velocities are nonrelativistic ($v_b\ll c$), the velocity can be expressed as a function of the potential over the anode and cathode as
\begin{displaymath}
v_b = \sqrt{\frac{2eU_{acc}}{m_e}}
\end{displaymath}
where $U_{acc}$ is the acceleration potential in volts, $e$ is the charge of an electron ($1.60\cdot10^{-19}$ C) and $m_e$ is the rest mass of an electron ($9.11\cdot 10^{-31}$\,kg). The velocity is given in m/s. Inserting this into equation \ref{eq:rho1} yields:
\begin{equation}
\rho = \frac{I}{\pi r_b^2} \sqrt{\frac{m_e}{2eU_{acc}}}
\end{equation}

Now consider that the electron beam is enclosed by a grounded conducting tube at a distance R from the center. This will be our reference potential ($U(R) = 0$).

From here on, $z$ will be used to denote the distance in the propagation direction of the electron beam, $r$ is the distance from the center of the beam in the plane perpendicular to $z$ and $\phi$ is the angle in this plane.\footnote{Do note I am using $r$ to denote the distance to the z-axis (a scalar) and \textbf{r} to denote a position (a vector).}

To get to the potential distribution, we apply Gauss' law \cite{GriffithsEM}. As a surface we take a cylinder enclosing the electron beam (that has a homogeneous charge density $\rho$), with length $l$ and radius $r$.
\begin{equation}
\int\textbf{E}\cdot d\mathbf{a} = E\cdot2\pi r l = \frac{Q_{enc}}{\epsilon_0} = \rho \frac{l\pi r_b^2}{\epsilon_0}
\end{equation}
Here $d\mathbf{a}$ denotes the unity vector perpendicular to the surface that we integrate over. The integral over $\textbf{E}\cdot d\mathbf{a}$ will give the flux of the electric field through this surface.
The two end facets of the cylinder are parallel to the field lines as we assume the charge distribution (the electron beam) to be infinitely long. Therefore, these two surfaces do not contribute to the integral.
The geometry described so far has radial symmetry, therefore the electric field will be in the radial direction.
\begin{equation}
\mathbf{E}(\mathbf{r}) = \rho \frac{r_b^2}{2\epsilon_0r}\hat{\mathbf{r}}
\end{equation}
Using the integral definition of the electric potential:
\begin{align}
U(r_b)-U(R) &= -\int_R^{r_b} \mathbf{E}\cdot d\mathbf{l} \nonumber \\
&= - \rho \frac{r_b^2}{2\epsilon_0} \int_R^{r_b} \frac{1}{r} dr
\end{align}
Solving the integral gives the potential at the beam edge ($U(r_b)$):
\begin{equation}
\label{eq:vbeamrho}
U(r_b) = - \rho \frac{r_b^2}{2\epsilon_0} \log\left(\frac{R}{r_b}\right)
\end{equation}
or, using equation \ref{eq:rho1}:
\begin{equation}
\label{eq:vbeamI}
U(r_b)=\frac{I}{2\pi\epsilon_0 c}\sqrt{\frac{m_e}{2U_{acc}}}\log\left(\frac{R}{r_b}\right)
\end{equation}
From here I will call $U(r_b)$ simply $U_b$. Although there also is a potential inside the electron beam, it is not of interest to our model.

\begin{figure}[h]
\centering
\input{Pictures/geometrytikz}
\caption{Geometry of the theoretical system.}
\label{fig:theorygeometry}
\end{figure}

\section{The sensing mechanism}
In the section above, a model was presented for the potential distribution outside the beam and within the grounded tube ($r_b\le r\le R$). If the beam is not in the center, this potential distribution changes. Measuring the potential distribution therefore gives information on the position of the beam. Before we deviate from the coaxial symmetry (we will do this in section \ref{sec:theory/deviating}), we will first look into a way to probe the potential using metal electrodes.
We first assume an infinitesimally thin metal plate, curved such that it coincides with an equipotential line. The plate would normally even out the potential as the free charges inside it can move freely to nullify any differences in potential across the metal. However, as the plate already is at an equipotential and it is thin, the effect of the plate on the field can be neglected. In this case, the potential of the plate will be the potential of the equipotential line it coincides with.
An example of a geometry where this is the case is given in figure \ref{fig:geofrontview}.
As electrons inside the beam can (in a quasi-static case) move freely under influence of external fields, we can treat the beam as a conducting rod that is lifted to the beam potential $U_b$.\red{REF?}
As the beam is treated as a conductor, the beam and electrode together can be treated as a capacitor (with capacitance $C_{bs}$).
The electrode itself is a distance $b$ from the grounded tube the beam is travelling through, so it has a capacitive connection to ground (with capacitance $C_g$). The equivalent circuit diagram corresponding to a rigid electron beam inside a hollow metallic tube with an electrode (on an equipotential line) is given in figure \ref{fig:capacities}.
\begin{figure}[H]
\centering
\begin{circuitikz}[american voltages]
\draw
(2,0) to [short] (0,0)
to [open, l^=$U_b$, o-o] (0,2)
to [short, o-] (2,2)
to [C, l_=$C_{bs}$] (2,1)
to [C, l_=$C_g$] (2,0)

(2,0) node [ground] {}

(2,1) to [short] (3,1)
to [open, l^=$U_s$, o-o] (3,0)
to [short] (2,0);
\end{circuitikz}
\caption{Equivalent electrical circuit diagram}
\label{fig:capacities}
\end{figure}

\begin{figure}[ht]
\centering
%\includegraphics[scale=.4]{frontview.PNG}
\input{Pictures/3dviewtikz}
\caption{geometry including sensor plate}
\label{fig:geofrontview}
\end{figure}

As the electrode is not grounded, the charge enclosed by the electrode is not screened. If the metals would be ideal, the plate would have an instantaneous induced dipole, moving all the charge $Q$ that's on the ``beam''-side to the ``ground'' side. If this is the case, the two capacitors (beam to sensor and sensor to ground) have an equal charge. However, they have different capacitance and therefore different potential differences.
In a realistic case, the induced dipole would be time-dependent and a fraction of the charge would not be transferred to the back of the electrode. We assume for this derivation that our metals are ideal.
To derive the sensor plate potential ($U_s$), we start by stating that the potential differences over the two capacities in figure \ref{fig:capacities} add up to $U_b$.
\begin{equation}
U_b = U_s + U_{bs}
\label{eq:potentialsum}
\end{equation}
It has been noted that the potential difference over the two capacitors is different because their capacitance is different. As the charges are equal, the ratio is given by
\begin{align}
Q &= C\cdot U = C_{bs} U_{bs} = C_g U_{sg} \nonumber \\
U_{bs} &= U_s\frac{C_g}{C_{bs}} \label{eq:Ubs1}
\end{align}
Combining equations \ref{eq:potentialsum} and \ref{eq:Ubs1} and solving for $U_s$ gives:
\begin{equation}
U_s = U_b \frac{C_{bs}}{C_{bs}+C_g}
\label{eq:vplatec}
\end{equation}
%where the indices $b$, $g$ and $bs$ are for the beam-to-ground, sensor-to-ground and beam-to-sensor respectively. $U_s$ is the potential of the sensor plate, $Q$ is the charge induced on the plate and $C_g$ and $C_{bs}$ are the capacities between the beam and the sensor plate and between the sensor plate and ground, respectively.

This derivation is valid for any electrode coinciding with a (part of an) equipotential surface placed in between a charge (distribution) and a grounded metal surface. By choosing the dimensions, the capacitance to the beam and ground can be varied.

If the electrode only covers a part of the equipotential surfaces, multiple identical electrodes can be used on the same equipotential. These then have equal capacitances and therefore equal potentials.
However, the capacitance of the electrodes to the beam changes when the beam is not centered. The pickup design presented in this report relies on this principle.
To know the output voltage of the electrodes, the capacities must be calculated.
First, consider the capacity of two concentric, cylindrical metal surfaces of length $l$, the outer one of radius $R_b$, the inner one of radius $R_a$. The capacity is then given by\cite{GriffithsEM}:
\begin{equation}
C = \frac{2\pi\epsilon_0l}{\ln \left(\frac{R_b}{R_a} \right)}
\label{eq:capconc}
\end{equation}
If now, instead of using a full circle, two half-circle surface are considered, they should have only half the capacity. This is of course neglecting the edges of the surfaces, but these edges are close to each other, effectively negating these effects. In the following analysis, the surfaces are considered to be almost-touching (neglecting edge effects) and to span an angle $\theta$ as can be seen in figure \ref{fig:geofrontview}.
We may calculate both the capacitance between the beam and the plates, $C_{bs}$ and between the plates and ground, $C_g$ in this way. In this case, naming the distance between the beam center and the plate $r_s$ and naming the radius of the grounded body $R$.
\begin{align}
C_{bs} &= \theta \frac{\epsilon_0l}{\ln\left(\frac{r_s}{r_b}\right)} \label{eq:capbeam}\\
C_g &= \theta \frac{\epsilon_0l}{\ln\left(\frac{R}{r_s}\right)}
\label{eq:capplate}
\end{align}
Here, $\theta$ is the angle covered by the sensor plate in the transverse plane. It can be seen that for a full circle ($\theta=2\pi$), the capacitance reduces to equation \ref{eq:capconc}.
Using equations \ref{eq:capbeam} and \ref{eq:capplate}, we can rewrite equation \ref{eq:vplatec} as a function of design parameters;
\begin{equation}
U_s = U_b \left(1-\frac{\ln\left(\frac{r_s}{r_b}\right)}{\ln\left(\frac{r_s}{r_b}\right)+\ln\left(\frac{R}{r_s}\right)} \right)
\label{eq:vsensor1}
\end{equation}
%U_s = U_b \frac{\theta\epsilon_0l \ln^{-1}\left(\frac{r_s}{r_b}\right)}{\theta\epsilon_0l \ln^{-1}\left(\frac{r_s}{r_b}\right)+\theta\epsilon_0l \ln^{-1}\left(\frac{R}{r_s}\right)}
Although equation \ref{eq:vsensor1} seems complete, the capacitance of the sensing electrode to ground is not well-defined. In the real system, there will be a number of possible components adding an unknown parasitic capacitance to ground (e.g. a wire or electronics). To take this into account, we add a parasitic capacitance term $C_{par}$ to the sensor's capacitance to ground.
\begin{equation*}
C_g^\prime = C_g + C_{par}
\end{equation*}
Exchanging the $C_g$ for $C_g^\prime$ in equation \ref{eq:vplatec}, the correct way for calculating the response is now a combination of equations \ref{eq:vplatec}, \ref{eq:capbeam} and \ref{eq:capplate}.

\subsection{Deviating from coaxial symmetry}
\label{sec:theory/deviating}
The plates are fixed in position with respect to the grounded tube surrounding them, making the capacity to ground $C_g$ a fixed value, even if the beam changes position. If the beam moves out of the center, only the value for $C_{bs}$ will change.
To calculate the potential response of the sensor plates to different beam positions, a model of this capacitance $C_{bs}$ must be constructed.
In the following derivation, the \textit{coaxial approximation} will be kept, but the radius of the sensing plates will be replaced for the distance between the beam center and the plate's inner surface. For a beam that is displaced from the center a distance $d$ upwards, this yiels:
\begin{align}
C_1 &= \theta \frac{\epsilon_0l}{\ln\left(\frac{r_s-d}{r_b}\right)} \\
C_2 &= \theta \frac{\epsilon_0l}{\ln\left(\frac{r_s+d}{r_b}\right)}
\end{align}
where $C_1$ and $C_2$ are the upper and lower electrodes' capacitance to the beam respectively. In total there are four electrodes, located in quadrants. These are numbered in the order ``up'', ``down'', ``left'' and ``right'' when looking \textit{toward} the electron gun. These numbers will now replace the subscripts of both $C_{bs}$ and $U_s$.
All plates have the same surface area with the same distance to the grounded tube surrounding them, so $C_g$ will be equal for all plates. Since the electronics will have very similar capacitances and the wires connecting the electrodes to the electronics will have a very small contribution to $C_{par}$, we assume $C_g^\prime$ to be equal for all four electrodes, independent of the position of the electron beam.
We then obtain (using a geometry constant $a=\theta\epsilon_0l$):
\begin{align}
U_1 &= U_b\frac{\frac{a}{\ln\left(\frac{r_s-d}{r_b}\right)}}{\frac{a}{\ln\left(\frac{r_s-d}{r_b}\right)}+C_g^\prime} \\
U_2 &= U_b\frac{\frac{a}{\ln\left(\frac{r_s+d}{r_b}\right)}}{\frac{a}{\ln\left(\frac{r_s+d}{r_b}\right)}+C_g^\prime} \label{eq:plateresponse}
\end{align}
In figure \ref{fig:plateresponse} the output of such a system where $U_b = 1\,$kV, $r_s=5\,$mm, $r_b=1\,$mm, $\theta=75\,\deg$ and $l=5\,$mm is shown as a function of the deviation of the electron beam position from the center.
\begin{figure}[hb!]
\centering
\includegraphics[width = 10cm]{plateresponse2v4.pdf}
\caption{Calculated potential of opposite electrodes as a function of vertical beam displacement.}
\label{fig:plateresponse}
\end{figure}

\newpage
\section{Signal processing and Linearisation}
The position of the beam center can be measured using the output of the electrodes. The geometry of the electrodes was optimized to maximize the signal on the electrodes along the axis of the displacement of the beam. This way, by analysing the difference in output of opposite plates, the position of the beam between them can be measured. Using two perpendicular sets of plates then gives the beam center position in the transverse plane.
In this section, the method of processing the signals from the electrodes to derive the position of the beam center.
\subsection{Measuring beam position}
To start off, only two plates are considered. The position is determined along the axis between the two electrodes. First, recall equation \ref{eq:vplatec} \footnote{The primes on $C_g$ will be omitted from this point. The parasitic capacitance is absorbed in $C_g$.}:
\begin{equation*}
U_i=\frac{C_i}{C_i+C_g}
\end{equation*}
We define the measured quantity as the difference between two opposite-plate potentials, divided over their sum ($\widetilde{\Delta U}$). This may seem arbitrary, but the reasons for this will become apparent at the end of this section.
\begin{align}
\widetilde{\Delta U} &= \frac{U_1-U_2}{U_1+U_2} \nonumber \\
&= \frac{\frac{C_1}{C_1+C_g}-\frac{C_2}{C_2+C_g}}{\frac{C_1}{C_1+C_g}+\frac{C_2}{C_2+C_g}} \nonumber \\
&= \frac{C_g(C_1-C_2)}{2C_1C_2+C_g(C_1+C_2)} \nonumber \\
\text{Assuming that $C_1,C_2 \ll C_g$:} \nonumber \\
&= \frac{C_1-C_2}{C_1+C_2} \label{eq:linearstep1}
\end{align}
This relates the electrodes' potentials to the capacities. It is worth noting that $\widetilde{\Delta U}$ does not depend on $C_g$. This implies that, as long as the parasitic capacitance is equal for all electrodes, the capacitance to ground has no effect on $\widetilde{\Delta U}$. Now to get to the actual displacement we combine equation \ref{eq:linearstep1} with \ref{eq:capplate}. % The variable $a$ is the same as in equation \ref{eq:plateresponse}. only necessary for intermediate step
\begin{align}
\widetilde{\Delta U} &= \frac{C_1-C_2}{C_1+C_2} \nonumber \\
%&= \frac{\frac{a}{\ln\left(\frac{r_s-d}{r_b}\right)}-\frac{a}{\ln\left(\frac{r_s+d}{r_b}\right)}}{\frac{a}{\ln\left(\frac{r_s-d}{r_b}\right)}+\frac{a}{\ln\left(\frac{r_s+d}{r_b}\right)}} \nonumber \\ % Leave out intermediate steps ?
&= \frac{\ln\left(\frac{r_s+d}{r_s-d}\right)}{\ln\left(\frac{(r_s+d)(r_s-d)}{r_b^2}\right)}\label{eq:linearstep2}
\end{align}
As we are intereseted in small values of $d$, a Taylor expansion is applied of \ref{eq:linearstep2} in $d$s:
\begin{equation}
\widetilde{\Delta U}=\frac{d}{r_s\ln\frac{r_s}{r_b}} + \frac{d^3}{r_s^3}\left(\frac{1}{2\ln\frac{r_s}{r_b}^2}+\frac{1}{3\ln\frac{r_s}{r_b}}\right)+O(\frac{d^5}{r^5}) \label{eq:linearstep3}
\end{equation}
The higher order terms scale with $d^n/r_s^n$. Therefore, as long as the displacement $d$ is smaller than $R-s$, the assumption that the higher order terms are negligible is valid. Consequently, for a small displacement $d$ compared to the distance to the sensing plates, $\widetilde{\Delta U}$ as seen in equation \ref{eq:linearstep3} will give a linear response to the deviation of the beam from the center.
\begin{equation}
d\approx\widetilde{\Delta U}r_s\ln\frac{r_s}{r_b}
\label{eq:reversecal}
\end{equation}
In figure \ref{fig:measd}, the result of equation \ref{eq:linearstep2} is plotted as a function of the displacement $d$. The parameters used here are the same as used in figure \ref{fig:plateresponse}.
It can be seen in this figure that, from a displacement of about 1\,mm, the higher orders of the Taylor expansion start being significant and $\widetilde{\Delta U}$ is no longer linear with respect to the displacement $d$. However, the coaxial approximation that was used to determine the potential of the electrodes also assumes $d$ to be small. Therefore, the exact displacement where the position monitor will stop behaving linearly must be experimentally verified.
\begin{figure}[h]
\centering
\includegraphics[width=10cm]{measureddeviationv3.pdf}
\caption{Theoretical calibration curve, $\widetilde{\Delta U}\,r_s\ln\frac{r_s}{r_b}$ as a function of $d$}
\label{fig:measd}
\end{figure}
In the final system four plates are placed in two pairs, such that they detect the position of the beam along two orthogonal axes. This way the position of the beam can be measured in the transverse plane.

\subsection{Determining beam voltage}
Although this measurement setup is apparently capable of determining the beam position, the beam current is also of interest as this is an indication of the electron source performance.
The beam voltage ($V_b$ in \ref{eq:vbeamI}) is a measure for the beam current and can be determined using the beam position monitor. To demonstrate this, we first consider the case where the beam is centered, $d=0$.
\begin{align}
\Sigma_{i=1}^4 U_i &= 4U_1 \nonumber\\
&= 4U_b\frac{C_1}{C_1+C_g} \nonumber\\
&= 4U_b\frac{a}{a+C_g\ln\left(\frac{r_s}{r_b}\right)}
\end{align}
In this equation, all values except $U_b$ are constant. Therefore, with the correct calibration, the beam voltage can readily be retrieved.
\begin{equation}
U_b =U_1\frac{a+C_g\ln\left(\frac{r_s}{r_b}\right)}{4a} \label{eq:beamvoltagemeas}
\end{equation}
The calibration is necessary, as the parasitic capacity $C_{par}$ inside $C_g$ is not exactly known.

Now consider a case where the beam is displaced in the y-direction. In order to improve readability, I will take a sidestep and show the Taylor expansion for two opposite plate potentials, considering only the affected electrodes.
\begin{align}
U_1 &= U_b \frac{a}{a+C_g\ln\left(\frac{r_s-d_y}{r_b}\right)} \nonumber \\
&= U_b \Bigg{(} \frac{a}{a+C_g\ln\left(\frac{r_s}{r_b}\right)} % zeroth taylor term
- \frac{aC_g}{\left( a+C_g \ln\left(\frac{r_s}{r_b}\right) \right)^2}\frac{d_y}{r_s} %first taylor term, note the minus sign
+ \frac{a C_g \left( a + C_g \ln\left( \frac{r_s}{r_b} \right) + 2C_g\right)}{ \left( a+C_g\ln\left(\frac{r_s}{r_b}\right) \right)^3}\frac{d_y^2}{r_s^2} %the second taylor term.
+O\left(\frac{d_y^3}{r_s^3}\right) \Bigg{)} \label{eq:VbU1}\\
U_2 &= U_b \frac{a}{a+C_g\ln\left(\frac{r_s+d_y}{r_b}\right)} \nonumber \\
&= U_b \Bigg{(} \frac{a}{a+C_g\ln\left(\frac{r_s}{r_b}\right)} % zeroth taylor term
+ \frac{aC_g}{\left( a+C_g \ln\left(\frac{r_s}{r_b}\right) \right)^2}\frac{d_y}{r_s} %first taylor term, note the minus sign
+ \frac{a C_g \left( a + C_g \ln\left( \frac{r_s}{r_b} \right) + 2C_g\right)}{ \left( a+C_g\ln\left(\frac{r_s}{r_b}\right) \right)^3}\frac{d_y^2}{r_s^2} %the second taylor term.
+O\left(\frac{d_y^3}{r_s^3}\right) \Bigg{)} \label{eq:VbU2}
\end{align}
Note that the odd terms of the Tayor expansion have opposite signs. Therefore, the even terms cancel out in the potential difference and the odd terms cancel out in the sum of potentials. A set of equations similar to \ref{eq:VbU1} and \ref{eq:VbU2} can be found for the horizontal electrodes' potentials, $U_3$ and $U_4$ but for a horizontal displacement $d_x$
Summing all four plate potentials then gives:
\begin{align*}
\Sigma_{i=1}^4 U_i &= U_b \Sigma_{i=1}^4 \frac{C_i}{C_i+C_g} \nonumber\\
&= U_b \left[ \frac{a}{a+C_g\ln\left(\frac{r_s-d_y}{r_b} \right)} + \frac{a}{a+C_g\ln\left(\frac{r_s+d_y}{r_b} \right)} + \frac{a}{a+C_g\ln\left(\frac{r_s+d_x}{r_b} \right)} + \frac{a}{a+C_g\ln\left(\frac{r_s-d_x}{r_b} \right)} \right]
\end{align*}
Expanding teach of the terms in the small displacements $d_x$ or $d_y$ gives:
\begin{align}
\Sigma_{i=1}^4 U_i = U_b \Bigg{(}
&\frac{4a}{a+C_g\ln\left(\frac{r_s}{r_b} \right)} \nonumber \\
+ &\frac{2aC_g\left(a+C_g\ln\left(\frac{r_s}{r_b} \right) +2C_g \right)}{\left(a+C_g\ln\left(\frac{r_s}{r_b} \right)\right)^3} \frac{d_x^2+d_y^2}{r_s^2} \nonumber \\
+ &O\left(\frac{d_x^4+d_y^4}{r_s^4} \right) \Bigg{)}
\label{eq:Vbtotal}
\end{align}
%+ &\frac{2aC_g\left(a+C_g\ln\left(\frac{r_s}{r_b} \right) +2C_g \right)}{\left(a+C_g\ln\left(\frac{r_s}{r_b} \right)\right)^3} \frac{d_y^2}{r_s^2} \\
For small displacements ($d_x,d_y \ll r_s$), the second order term can also be neglected and \ref{eq:Vbtotal} reduces to equation \ref{eq:beamvoltagemeas}. Therefore, for sufficiently small displacements, equation \ref{eq:beamvoltagemeas} still holds.
