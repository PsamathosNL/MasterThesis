\chapter{Conclusion and Outlook}
The goal of this research was to find a method to align the electron beam in the photonic-Free Electron Laser.
The DC-coupled Capacitive pickup design presented in chapter \ref{sec:design} has been built by the TCO of the University of Twente and its quality has been tested. 
The range along the beam path at which the lateral position could be measured was not the entire length of the photonic crystal (465\,mm), but only around (80\,mm). In this case, a displacement of down to \SI{242}{\micro\meter} is necessary. However, the requested accuracy was already better than this.
The accuracy of this beam position monitor was found to be \red{some number}, significantly better than the requested accuracy of \SI{100}{\micro\meter}. The accuracy was measured outside of the vacuum chamber using a metallic rod as a model electron beam.
The other demands presented in chapter \ref{sec:design} were also met. 
The vacuum chamber took only a week of baking-out to reach a pressure of $< 2\cdot 10^{-8}$\,mbar. Even though there were some electronics with plastic resin surfaces, some bare, untreated copper and trace amounts of lead solder the pumping down time was satisfactory.
With a pressure feedback control system, this might have been significantly reduced even further.
The guiding system was placed into the set-up without problems and the tolerances for placing the alignment pins (\red{UNKNOWN}) were reached.