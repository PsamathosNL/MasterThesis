\chapter{Simulations on Location errors}
\label{sec:alesresults}
%This appendix will present two simulations that Ale did, one with a misaligned electron beam (slightly displaced) and one with a tilted magnetic field.
%The goal here is to show that the simulations give non-physical results if the electron beam is close to a perfect electrical conductor.
In all previous simulations, the electron beam was assumed to be located exactly in the middle. However, an electron beam is never perfectly aligned \red{citation needed}, there is always some difference between the position and direction of the beam and those of the Photonic Crystal.
In order to determine the influence of this misalignment, a few extra simulations were carried out.

In these simulations the positions of the ``macro-particles'' (the particles representing the electrons) are recorded at different moments in time. Using this approach, the electron trajectories can be mapped and if there are anomalies, these can easily be found.

In the first simulation, the beam source was moved to the \red{left? right? up?} a distance of \red{something}.
\red{add pictures of the beam paths at the start and end of the simulation, showing the wiggles at the end (an effect of the gain)}
\red{conclusions from this simulation}

\red{second simulation: diagonal B-field}
In the second simulation, the beam source was kept at the center, but the magnetic field focussing the beam was tilted \red{how much?}.
\red{again, add pictures}
