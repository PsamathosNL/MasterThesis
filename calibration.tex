\chapter{Calibration}
% kom terug op alle dingen die je zegt te zullen onderzoeken
% zener diode verlaagt ingangsimpedantie
% meting met kabels?
% meting u1, u2 los als functie van afstand, naast theorie plotten
\label{sec:calibration}
The design described in the previous chapter was built. This section describes the characterisation and calibration of the device. This is carried out to determine whether the requirements postulated in the introduction were met. The comparison of the results with the requirements will be in the last part of this chapter.

Ideally, the position monitor system would be calibrated by placing it at a known position, near an electron beam source (with the electron beam going through the sensor, and varying the relative position while recording the response. There are a number of reasons why this approach is not feasible. 
First, the position of an electron beam has to be measured by a secondary system, so another position monitor would have to be present for calibration. As the energy density of this electron beam is so high, typical intercepting sensors are not possible. Non-intercepting sensors are typically used for high repetition rate, pulsed electron sources. As the readout electronics would have to be completely different when using these sensors with a quasi-continuous electron beam, these sensors can not be used for calibration.
Second, in order to have a proper calibration, the BPM response must be recorded for a range of different positions. This requires either a beam line with steering capabilities or a translation system in the lateral direction, and neither of these possibilities is available in the current experimental setup.
Third and last, the entire system, now consisting of 2 beam position monitors, a translation stage or beam steering and an electron beam source, would have to fit inside a vacuum chamber. This means all parts have to be ultra-high vacuum compatible, compact and accurately moveable up to the accuracy that is required of the position monitor, in our case \SI{100}{\micro\meter}.

All the above is challenging, but possible. A thermionic cathode from a CRT TV could be used to guide a low-current electron beam through different lateral positions within the BPM. The beam position could be measured after the BPM using a phosphor screen, providing a secondary measurement and thus a proper calibration method. However, the beam produced by the CRT type of electron gun has completely different current (2 orders of magnitude lower) and beam energy (unknown) \red{citation needed}.
Also, a more similar electron gun could be used and an in-vacuum translation stage could be utilized to change the relative position. However, in this case, a proper secondary measurement is not feasible.
Because of these problems, combined with the cost and production time of such a setup, the consideration was made not to use an electron beam to calibrate this beam position monitor. Instead, the electron beam was simulated using a current through brass rod of the same diameter.
The beam position monitor is translated laterally compared to the brass rod and the response is recorded. Using this method, the characterisation and calibration has been done completely outside the vacuum with more flexibility to vary electron beam parameters while keeping a good approximation of the real system. 

In the remainder of this chapter, the translation set-up is described, followed by the excitation circuit and the method of signal acquisition. After this, the typical characteristics of the signals from the sensor are considered and the results of the calibration are presented. 

\section{Calibration method}
In order to verify the expected results with reality, the following calibration method was devised. 
A metal rod was used to simulate the electron beam. It was brought to a voltage equal to the beam potential $U_b$ described in equation \ref{eq:vbeamrho} and the current was varied to accommodate a range of currents expected from the electron gun. %Although the current has no direct effect on the measured values, the sudden change in current creates a large induction that introduces noise in the measurement. Other position monitor designs \textcolor{red}{REF!} rely on this effect specifically as their method of operation. 
The beam potential $U_b$ was also varied to determine any dependencies of the position monitor's behaviour on the charge density of the beam. Then, the response of the four electrodes was measured for a range of positions.

\subsection{Translation}
\label{sec:calibration/translation}
The BPM must be accurately moved with respect to the rod. Furthermore, the rotation of the BPM with respect to the rod axis must be controlled in order for the rod and BPM axes to be aligned.
In the setup shown in figure \ref{fig:schem_translation}, the BPM was held by a metal holder that was firmly placed on a 2-axis translation stage. This allowed the monitor to be moved in the transverse plane with up to \SI{10}{\micro\meter} accuracy. The rod was kept in a fixed position.
The BPM was secured by two sets (one at the entry point of the beam into the BPM and on at the exit) of four screws, allowing control over the rotation of the sensor body. The BPM could be moved until the rod was touching the isolation of the sensing electrodes. 
In order to determine the center position, the BPM was moved to the extremes\footnote{As the rod could bend, the extremes were taken to be ``the point at which changing the position no longer changed the output of the electrodes''.} in the Y-direction, and the positions were noted. The average of the two were taken to be the center and the BPM was then positioned in this position. Then the X-axis extremes were noted and the process was repeated until the values for the center positions converged. This gave a measurement of the center position with \SI{150}{\micro\meter} precision.
\begin{figure}[H]
\centering
\input{Pictures/setuptikz}
\caption{A schematic view of the translation setup with Beam Position Monitor (BPM)}
\label{fig:schem_translation}
\end{figure}

\subsection{Excitation system}
The calibration method is designed to simulate a range of possible values for the charge density and electron current. 
As mentioned before, the electron beam was mimicked by a metal rod of the same diameter (2\,mm). This rod was suddenly brought to the voltage $U_b$, that corresponds to the voltage at the edge of the electron beam. Also, a current equal to the beam current was supplied through the rod.
After a pulse time corresponding to one electron beam pulse (\SI{10}{\micro\second} the voltage on and current through the rod were dropped to zero. This was achieved using the setup seen in figure \ref{fig:circuitsetup}.

\begin{figure}[h]
\centering
\input{Pictures/electrictikz}
\caption{Circuit diagram of the system used to produce the excitation current through the rod.}
\label{fig:circuitsetup}
\end{figure}

In the setup, a capacitor $C$ (\SI{1000}{\micro\farad}) was charged to the beam voltage $U_b$ (between 600\,V and 1200\,V) using a high-voltage DC power supply via a charging resistor $R_\text{charge}$ (14\,\k$\Omega$) to limit the current drawn from the power supply.
A high-voltage switch (Behlke model \red{xx}) was used to to discharge the capacitor via the rod and a discharging resistor $R$. The resistor $R$ (between 300 and 1200\,$\Omega$) determines the current for a set beam voltage.
The characteristic RC-time of the discharging system is so large that the voltage drop over C during the current pulse of \SI{10}{\micro\second} can be neglected.
The rod was 20\,cm long to avoid any end effects or effects of the wires connected to it. The resistance of the rod (about 12.6\,m$\Omega$) was neglected.

% some part about excitation pulse maybe?
%
\section{Electrode response}
\label{sec:calibration/signal}
As discussed before, any charge (density) inside the BPM induces an electric potential at the electrodes. There are a number of characteristics and distortions in the potential of the plates that are discussed in this section.
In order to measure the potential, the electrode needs to be connected to a measurement device, which results in charge flowing through the device and consequently a drop in the potential of the electrodes.
The time it takes for the potential of the electrode to drop to near-zero is approximately three times the characteristic RC-time where R is the input impedance of the measurement device and C is the capacitance of the pickup electrode to ground. In our system, the capacitance to ground is about 10\,pF. Using a 50\,$\Omega$ measurement device, the potential drops to near-zero in 1.5\,ns.
The electrode capacitance can not be changed and externally adding a sufficient parasitic capacitance to increase it would drastically lower the electrical potential. Therefore, the input impedance of the measurement system must be increased.
A buffer circuit as described in section \ref{sec:design/circuit} was used to limit the current drawn by measurement devices. The circuit as a whole has an input impedance of around 10\,M$\Omega$, increasing the RC-time to around \SI{10}{\micro\second} regardless of the impedance of the measurement device used.
Altough the potential drops significantly during a single pulse of the pFEL electron gun, this is slow enough that information can still be extracted.

The buffer circuit does introduce a distortion to the signal. Since the buffer electronics have a limited bandwidth, the fast rise in potential at the input of the circuit creates a damped oscillation at the output.

%The sudden onset of the beam current induces a large change in the created magnetic field, in turn creating a short-lived current near $t=0$ and near the end of the electron beam ($t=\tau_b$). This has not been taken into account in the theory as it is not the method this position monitor has been built to use. However, as the isolation circuits have limited bandwidths, The sudden impulse creates an oscillation in the output signal.
Another distortion of the signal is the offset caused by the electronics. The amplifier that was used has no internal balancing mechanism. This causes the output signal to have a non-zero value ($U_{offset}$)when the input signal is zero. This could be counteracted by using the offset pins on the LT-1122 \cite{LT1122}, but this option was not pursued as this would nearly double the required cables through the vacuum tube. Instead this offset was compensated for in the post-processing of the measured signals.

%\begin{figure}[H]
%\centering
%\input{Pictures/signaltikz2}
%\includegraphics[width=300pt]{signalex.pdf}
%\caption{\red{CHANGE X-axis to have beam arrive at $t=0$}A sketch showing the signal characteristics; non-zero voltage at $t<0$, then rapidly fluctuating signal followed by exponential decay and a sudden drop at $t=\tau_b$ (not to scale).}
%\label{fig:signalex}
%\end{figure}

% Introduce how the signals are handled. How do I go from a temporal signal to a number?
% I use an algorithm that measures the offset before the peak, neglects the peak itself and averages over the time afterwards. However, this is mostly empirical. I know it works way better than measuring "by hand" but it's hard to say why exactly without presenting a lot of data, some of which is not saved anywhere.
In order to negate distortions, the potential of an electrode is averaged over time. This average is started after the initial peak at beam arrival and stopped as soon as the exponential decay becomes apparent. The exact time values are determined empirically. The measured offset (time-average of the potential before beam arrival) is subtracted from the measured average potential to give the actual result.

\section{Results}
In this section, the results of the calibration are presented. First, a measurement was carried out to determine the extent of the assumptions made in the theoretical model. Second, the response of the system to different displacements was measured at various values for $U_b$ and $I$ (through $R$).

\subsection{Extent of assumptions}
The most important assumption was that the displacement $d$ was small. The coaxial approximation and the Taylor approximation were both depending on this. The displacement was increased as much as the set-up allowed and the potential of the electrodes was recorded at various positions. As only the extent of the linearity of the difference over sum is of interest here, no normalization was carried out.
\red{FIRST ADD MEASUREMENTS OF THE TWO ELECTRODES, THEN ADD THE DOS}.
\begin{figure}[h]
\centering
\includegraphics[width=10cm]{linearitymeasref.pdf}
\caption{The measured displacement compared to the set position. A linear fit was made to extract the constant of proportionality.}
\label{fig:linearitymeas}
\end{figure}

As can be seen when comparing figures \ref{fig:linearitymeas} and \ref{fig:measd},the range where the sensor response is linear is found to be significantly larger than expected. This can be explained by the fact that two approximations are breaking down. The higher order terms of the Taylor approximation in equation \ref{eq:linearstep3} no longer are negligible. Taking these terms into account would negatively influence the signal output according to figure \ref{fig:measd}. What has not been incorporated in that figure is the breakdown of the coaxial approximation. In the extreme case of the beam being very close to one of the plates, the coaxial approximation assumes there are 2 small coaxial cylinder surfaces. In reality, there are a cylindrical part next to a nearly straight wall. The latter has a higher capacitance, so $C_{bs}$ is higher than expected. According to equation \ref{eq:vplatec}, increasing $C_{bs}$ results in an increase in the potential response. When considering figure \ref{fig:linearitymeas} it can be concluded that the decrease of the plate potential due to the higher order terms in the Taylor expansion and the increase of the potential due to the increased capacitance cancel each other out almost completely along the measurable range.

\subsection{Numerical results}
The BPM was calibrated for 9 different combinations of beam voltage and current. Each calibration consisted of measurements of the plate potential at 81 different positions. These 81 points correspond to a matrix of 9 by 9 positions in a 1.4x1.4 mm square of positions relative to the electron beam. Close to the center a small step size of \mm{0.1} was chosen for high accuracy. The outer two series of steps were \mm{0.25} in order to cover a larger distance.
At each position, the signals from all four plates were measured by averaging the portion of the signal after the initial peak but before the onset of the exponential decay. The difference-over-sum was calculated and the result is shown in figure \ref{fig:difovsum}.
\begin{figure}[h]
 \centering
 \includegraphics[width=300pt]{difovsum.pdf}
 \caption{A graph showing the (non-normalized) difference-over-sum of two opposite plate potentials at various positions.}
 \label{fig:difovsum}
\end{figure}

The linear dependence of the difference-over-sum on the position can readily be seen in figure \ref{fig:difovsum}. To obtain a normalized value, the constant of proportionality (CoP) was derived by fitting a linear surface.
A number of measurements were carried out to check if the CoP depends on different beam parameters such as the charge density (through the beam voltage) or total beam current. Although this is not seen in the theoretical model, a dependence was found on the beam voltage $U_b$.
\red{This is thought to be caused by the method of measuring the plate voltage. The actual voltage is too noisy to measure, so the signal is measured later in time by averaging over an interval. The signal decays over time so the average is lower in voltage. This voltage drop is due to exponential decay (a capacitor discharging) and so, the higher the signal originally was, the more it drops.}

When measuring, the sum of the signals is used to determine the beam voltage. In this case, zero displacement is assumed. The CoP's for different beam voltages is then looked up and the correct value is interpolated. This value for the CoP is then used to determine the displacement of the beam.

In figure \ref{fig:calfig}, the red circles are measured positions and the blue ones are set positions. The center position was determined as described in section \ref{sec:calibration/translation}. The calibration was carried out using a set of 9 measurements like the one shown and finding the parameters that minimized the errors.

\begin{figure}[H]
\centering
\includegraphics[width=300pt]{calibrationfig.pdf}
\caption{The set (blue) and measured (red) positions after calibration.}
\label{fig:calfig}
\end{figure}
