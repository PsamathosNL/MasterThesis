\chapter{Introduction}
\label{sec:introduction}
%Probleemschets
% Introduction to pFEL operation and setup
% 
% pFEL is theoretisch cool
In novel free electron laser designs (so-called photonic free-electron lasers), electrons stream through a photonic crystal. Interaction of the electrons with a Bloch eigenmode of the photonic crystal results in bunching of the electrons. This bunching results in coherent amplification of the Bloch eigenmode \cite{Denis}.
The bunching, and therefore the gain of the laser, strongly depends on the properties of the electron beam(s).
This report describes a sensor system that is able to measure the transverse location and charge density of the electron beam and consequently will be used to characterize the alignment of the electron beam used in a photonic free-electron laser.
%
%, a high gain and continuous tuning have been presented. \cite{Denis} Such systems require a high charge-density quasi-continuous electron beam to be shot through a photonic crystal (PC). 
%In order to achieve a good approximation of the designs and simulations presented in these works, the electron beam must be localized and any deviation from the simulations in this regime must be eliminated up to small deviations (See appendix \ref{sec:alesresults}). However, the set-up has no method of locating the electron beam built-in.

\section{The photonic Free Electron Laser}
%CFEL --> problems with high power beams --> pFEL multiple beams --> diagnostic is missing --> This report describes solution
Cherenkov Free Electron Lasers (CFEL's) have been a field of study for a long time \cite{garate1987cherenkov}/cite{walsh1977cerenkov}. However, generally the output power is limited due to the nature of the Cherenkov radiation.
In order to increase the power radiated, the electron current density must be increased. Consequently a high current density leads to a strong coulomb repulsion between electrons which makes it difficult to generate and transport such a beam.

In order to overcome these problems, the photonic Free-Electron laser was developed. Here, the homogeneous dielectric medium is replaced by a metallic photonic crystal (PC)\footnote{Although a dielectric PC could also be used, but the higher thermal conductivity and easy charge transport make metal a more convenient choice.}.
The merits of the PC as a Cherenkov medium have been shown in simulations, possibly allowing a continuously tunable source with high output power.
The PC also allows multiple electron beams to contribute mostly to the same eigenmode in the photonic crystal, effectively negating the biggest drawback of the CFEL.\cite{DenisMultiBeam}.
A proof-of-principle set-up has been built \cite{Denis} but e-beam diagnostics were not implemented. The work presented in this report remedies this problem.
In this set-up, the frequency was expected to be 16\,GHz. The photonic crystal period was 4.2\,mm. The electron beam was accelerated with voltages ranging from around 10\,kV up to 15\,kV and the beam current was measured to be around 1\,A.

\section{e-beam diagnostic methods}
Several methods exist to detect properties of the electron beam. The properties that are most interesting to the functionality of the pFEL are the transverse position, charge (or current) density and electron speed. The electron speed can readily be derived from the acceleration voltage (which can be measured externally), so a method should cover the transverse position and the charge density. 
Here, we briefly present advantages and disadvantages of some present techniques.

\paragraph{Optical methods}
In certain materials, the kinetic energy of the electrons is absorbed and converted to optical photons by a variety of effects \cite{leo1994techniques}. These can then be imaged and thus the beam profile can be determined. A well known example of a material that has this property is phosphor. It is used to build the CRT-type screens where electrons are scanned over a glass surface that is coated with phosphor.
Another, more sturdy material that has this property is a Cerium-doped YAG crystal (Ce:YAG).
However, although the kinetic energy is not very high in this system (15\,keV), the current is relatively high at up to \SI{2}{\ampere}. This current is spread over a beam with a radius of \mm{1}, giving the surface it hits a power density of around $10\,\text{kW\,mm}^{-1}$
This is significantly more than any of the optical methods are made for. The surface melts and partly evaporates shortly after exposure to this electron beam. 
The effects can be seen in figure \ref{fig:brokenyag}. Shown are the remains of a Ce:YAG screen with a layer of titanium to block the glow from the hot cathode. The metal has been molten over a large part of the surface and the YAG crystal has clearly visible damage where the beam hit. 
\begin{figure}[h]
 \centering
 \includegraphics[width=200pt]{brokenYAG.png}
 \caption{A photograph of a Ce:YAG screen that was used to determine the beam position. The scratches were caused by improper handling. At the top the metal is thought to have molten and formed a liquid drop. At the center a colourful corona is visible and a circular hole can clearly be distinguished.}
 \label{fig:brokenyag}
\end{figure}

In order to allow for optical methods to be used, the energy density would have to be lowered by at least 4 orders of magnitude. 
This would require either a shorter pulse time, a lower beam current (charge density) or a lower acceleration potential.
Due to the electronics in the electron gun control, the pulse time can not be made shorter than \SI{1}{\micro\second} which does not solve the problem. 
%de snelheid of ladingsdichtheid aanpassen zorgt ervoor dat het magneetveld niet meer gematcht is. Stel dat we dit observeren en corrigeren bij een aangepaste magneetveld/ladingsdichtheid wil dat niet zeggen dat de nominale magneetveld/ladingsdichtheid combinatie ook goed werkt.
Changing the charge density or the acceleration potential creates a mismatch between the charge density, acceleration potential and magnetic field. If this mismatch would be corrected, the diagnostics would not be able to guarantee that the nominal acceleration voltage and charge density would be matched.
Furthermore, the electron gun has a certain set voltage range where the beam is well-defined, so the acceleration potential can not be lowered significantly.
Hence, the energy deposition by the electrons can not be decreased to a level suitable for a screen.

\paragraph{Electrical methods}
Another method to locate an electron beam is to use a faraday cup. This is a cylinder with one open end and a ``target'' in the center. The target is connected to electrical ground via a current measurement device. 
The electron beam hits the target and the current to ground is measured. If the cup consists of multiple segments, the current profile over different segments directly translates to a beam profile.
This enables a faraday cup to be used as a beam profile measurement device, giving information on the current distribution, total current and location of the beam.
However, our simulations (see appendix \ref{sec:heatsims}) have shown that when the electron beam is normally incident on the surface of the target, the surface is heated up to above the melting point of any metal\footnote{except Adamantium and Mithril which are not readily available at this time.}.

Another option is a beam position monitor (BPM). This is a device generally used for pulsed ion or electron beams. It consists of a set of electrodes (metal plates) parallel to the beam path. A passing electron pulse induces a voltage on the electrodes. The voltages of four electrodes distributed over 360 degrees are processed to produce signals that are proportional to the diplacement of the center-of-charge from the axis of the sensor and proportional to the beam current (or pulse charge).
As BPM's rely on capacitive coupling and not on impact measurements, the beam can continue undisturbed. Therefore, BPM's are so-called non-intercepting detectors.

Although this is a viable method for measuring high-power electron beams, it is generally only used for pulsed beams, while the pFEL relies on a quasi-static beam.

In this report the design of a BPM optimized for operation in a pFEL is explained.
By utilizing the principles of a set of capacitive pick-up electrodes, the beam is not intercepted. This is a huge advantage as the detecting method is no longer limiting the current density of the beam as is the case with intercepting methods (screens, faraday cup) where the energy from the electron beam is turned to destructive heat. 
In the following chapters, a theoretical background of the principles of BPM's will be explained, followed by the actual design of this specific BPM. Also, a chapter is included explaining the calibration of the BPM.

%korte samenvatting: we hebben X gedaan
%  inschatting van uitkomst/ theorie
%  bouwen
%  testen/meten
